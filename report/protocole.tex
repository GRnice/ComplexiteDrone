\documentclass[a4paper,12pt]{article}
\usepackage[T1]{fontenc}
\usepackage[french]{babel}
\usepackage[utf8]{inputenc}
\usepackage{hyperref}
\usepackage{graphicx}
\usepackage{tkz-graph}
\usepackage{multirow}
\usepackage{array}
\usepackage{listings}
\newcolumntype{P}[1]{>{\centering\arraybackslash}p{#1}}
\newcolumntype{M}[1]{>{\centering\arraybackslash}m{#1}}
\usepackage{tikz}
\usetikzlibrary{arrows,calc,positioning}
\tikzstyle{intt}=[draw,text centered,minimum size=6em,text width=5.25cm,text height=0.34cm]
\tikzstyle{intl}=[draw,text centered,minimum size=2em,text width=2.75cm,text height=0.34cm]
\tikzstyle{int}=[draw,minimum size=2.5em,text centered,text width=3.5cm]
\tikzstyle{intg}=[draw,minimum size=3em,text centered,text width=6.cm]
\tikzstyle{sum}=[draw,shape=circle,inner sep=2pt,text centered,node distance=3.5cm]
\tikzstyle{summ}=[drawshape=circle,inner sep=4pt,text centered,node distance=3.cm]

\lstset{
  language=C,                % choose the language of the code
  numbers=left,                   % where to put the line-numbers
  stepnumber=1,                   % the step between two line-numbers.
  numbersep=5pt,                  % how far the line-numbers are from the code
  backgroundcolor=\color{white},  % choose the background color. You must add \usepackage{color}
  showspaces=false,               % show spaces adding particular underscores
  showstringspaces=false,         % underline spaces within strings
  showtabs=false,                 % show tabs within strings adding particular underscores
  tabsize=2,                      % sets default tabsize to 2 spaces
  captionpos=b,                   % sets the caption-position to bottom
  breaklines=true,                % sets automatic line breaking
  breakatwhitespace=true,         % sets if automatic breaks should only happen at whitespace
  title=\lstname,                 % show the filename of files included with \lstinputlisting;
}

\title{%
        Protocole de déplacement pour robots à base d'Arduino\\
        \large Complexité\\
        \href{https://github.com/GRnice/ComplexiteDrone}{https://github.com/GRnice/ComplexiteDrone}
}

\author{%
        M1 IFI/RIF
}


\begin{document}
\maketitle

%\tableofcontents
\begin{abstract}
        Ce document décrit le travail effectué tant au niveau de la conception de la formation et du protocole de déplacements des robots que de l'implémentation. Il décrit l'évolution possible du projet ainsi que les personnes ayant participés à sa réalisation.
\end{abstract}

\newpage

\section{Protocole}\label{protocole}
\subsection{Initialisation}
La phase d'initialisation, à l'allumage des robots, se déroule en trois étapes :
\begin{itemize}
        \item Tirage d'un identifiant unique ;
        \item Demander s'il existe déjà un chef, s'il n'y en a pas, le robot se considère comme étant le chef ;
        \item Envoyer son identifiant unique au chef ;
        \item \textit{Facultatif} : Vérifier s'il existe plusieurs chefs, si oui celui avec l'identifiant le plus grand garde son statut de chef.
\end{itemize}
Pour finir le chef envoie à chaque robot une paire de nombres composée de l'identifiant du robot et de sa position dans la formation.
\subsection{Formation}
Les robots se positionnent en triangle, avec le chef en tête et deux esclaves derrière lui.\\
Chaque esclave est suivit par deux esclaves jusqu'à la fin de la formation.\\
Les robots sont séparés par une distance de $50 \pm 10\%$ cm.
\begin{figure}[!h]
        \centering
        \begin{tikzpicture}
                \GraphInit[vstyle=Empty]
                \Vertex[x=1, y=-2]{C}
                \Vertex[x=0,y=-3]{E1}
                \Vertex[x=2,y=-3]{E2}
                \Vertex[x=-1,y=-4]{E3}
                \Vertex[x=1,y=-4]{E4}
                \Vertex[x=3,y=-4]{E5}
                \Vertex[x=-2,y=-5]{E6}
                \Vertex[x=0,y=-5]{E7}
                \Vertex[x=2,y=-5]{E8}
                \Vertex[x=4,y=-5]{E9}
                \Edges(E1,C,E2)
                \Edges(E3,E1,E4,E2,E5)
                \Edges(E6,E3,E7,E4,E8,E5,E9)
        \end{tikzpicture}
        \caption{Formation avec dix robots}\label{fig:Form}
\end{figure}

\subsection{Mouvement}
Lors du déplacement, le robot avec le rôle de chef respecte les conditions suivantes :
\begin{itemize}
        \item Avancer en ligne droite ;
        \item S'il détecte un obstacle, l'éviter ;
        \item Communiquer ses déplacements aux autres robots.
\end{itemize}
Le reste des robots obéissent aux règles ci-dessous :
\begin{itemize}
        \item Écouter les ordres de déplacements du chef ;
        \item Avancer en fonction de l'ordre reçu et de leur position dans la formation
\end{itemize}

\subsection{Format de communication}

\begin{figure}[!h]
        \centering
        \begin{tabular}{|l|l|l|}
                \hline
                num\_de\_drone & type\_de\_message & contenu \\ \hline
        \end{tabular}
        \caption{Modèle de message}\label{fig:Mesg}
\end{figure}
Le format de communication se présente comme une trame qui possède une partie fixe et variable avec les éléments suivants :
\begin{itemize}
        \item \textit{num\_de\_drone}, l'identifiant du robot ;
        \item  \textit{type\_de\_message}, le type de message envoyé ;
\item \textit{contenu}, le contenu du message.
\end{itemize}

\newpage
\subsection{Déroulement des communications}
Le schéma des communications commence toujours par un robot qui envoie "chef", ensuite en fonction de la réponse qu'il reçoit, les communications empruntent une branche de l'arbre des communications.
Les cases bleus sont les messages envoyés et les cases rouges sont les messages reçu par le robot.
\begin{figure}[!h]
        \centering
        \begin{tikzpicture}[
                >=latex',
                auto
                ]
                \node [intg] (kp)  [draw=blue] {chef};
                \node [int]  (ki1) [node distance=1.5cm and -1cm,draw=red,below left=of kp] {Id};
                \node [int]  (ki11) [node distance=1.5cm and -1cm,draw=blue,below =of ki1] {UId};
                \node [int]  (ki12) [node distance=1.5cm and -1cm,draw=red,below =of ki11] {(UId,Num)};
                \node [int]  (ki2) [node distance=1.5cm and -1cm,draw=red,below right=of kp,label=right:\& timeout > 10sec] {0};
                eroiejg
                \node [int]  (ki21) [node distance=1.5cm and -1cm,draw=red,below=of ki2] {chef};
                \node [int]  (ki22) [node distance=1.5cm and -1cm,draw=blue,below=of ki21, ] {Id};
                \node [int]  (ki23) [node distance=1.5cm and -1cm,draw=red,below=of ki22] {UId};
                \node [int]  (ki24) [node distance=1.5cm and -1cm,draw=blue,below=of ki23] {(UId,Num)};
                \node [int]  (ki3)  [node distance=1.5cm and -1cm,draw=blue,below left=of ki24] {direction};
                \node [int]  (ki4)  [node distance=1.5cm and -1cm,draw=blue,below right=of ki24] {(vitesse, temps)};
                \draw[->] (kp) -- ($(kp.south)+(0,-0.75)$) -| (ki1);
                \draw[->] (kp) -- ($(kp.south)+(0,-0.75)$) -| (ki2);
                \draw[->] (ki1) -- (ki11);
                \draw[->] (ki11) -- (ki12);
                \draw[->] (ki2) -- (ki21);
                \draw[->] (ki21) -- (ki22);
                \draw[->] (ki22) -- (ki23);
                \draw[->] (ki23) -- (ki24);
                \draw[->] (ki24) -| ($(ki24.east)+(1,1)$) |- (ki21);
                \node [] [node distance=1cm, right=of ki22]{\& timeout > 1min};
                \draw[->] (ki24) -- ($(ki24.south)+(0,-0.75)$) -| (ki3);
                \draw[->] (ki24) -- ($(ki24.south)+(0,-0.75)$) -| (ki4);

                \begin{scope}[
                        node distance=1em and 1em,
                        ]
                        \node[draw=red,right=-8cm of ki24.west, text opacity=0, label=right:message reçu] (H) {sdgr\newline dfsf};
                        \node[draw=blue,below=of H, text opacity=0, label=right:message envoyé] (A1) {sdgr\newline dfsf};
                        \path let
                        \p2=(H.west),
                        \p1=(A1.east)
                        in
                        node[align=center,text width=\x1-\x2,anchor=south west,inner sep=0pt]
                        (title)
                        at ([yshift=5pt]H.north west)
                        {Légende};
                \end{scope}
        \end{tikzpicture}
        \caption{Déroulement des communications du point de vue d'un robot}
        \label{fig:Comm}
\end{figure}

\section{Implémentation}
\subsection{Machine à états}
Le code à été implémenté sous forme de machine à états, avec les trois états suivants :
\begin{itemize}
        \item ETAT\_INIT ;
        \item ETAT\_GO ;
        \item ETAT\_STOP.
\end{itemize}

Le robot démarre dans l'état \textit{ETAT\_INIT}, dans lequel il participe à l'élection du chef. À la fin de la phase d'initialisation, il passe dans l'état \textit{ETAT\_GO}.
Il est important de noter que la phase d'initialisation du chef est la plus longue, en effet il partira lorsque l'ensemble des suiveurs sera pret et en position, à une distance minimale de leur prédécesseur.

Dès que le chef détecte un obstacle, il envoie un message \textit{stop}, en attendant qu'il choisisse la manière de l'éviter, les suiveurs passent à l'état \textit{ETAT\_STOP} et leur moteur s'arrête.
Quand le chef trouve une direction libre, il envoie un message aux suiveurs pour qu'ils passent à l'état \textit{ETAT\_GO}.

La procédure \textit{go} est elle aussi implémentée à partir d'une machine à états. Ainsi il n'est plus nécessaire pour le chef d'indiquer à chaque tour de boucle aux suiveurs de rester dans l'état \textit{ETAT\_GO} et de continuer à avancer. Cela libère donc du temps de communication et réduit la redondance des messages.

\subsection{Messages}
Les messages sont envoyés uniquement par le maître (ou chef), seul les esclaves (ou suiveurs) peuvent donc les écouter. Ils sont dans une logique simplex avec un mode de communication en broadcast.

\subsection{Maître}
Le maître peut effectuer deux actions :
\begin{itemize}
        \item Envoyer un message à chaque fois qu'il effectue un mouvement, c'est-à-dire à chqque tour de boucle ;
        \item Vérifier la réception d'un message à chaque tour de boucle.
\end{itemize}

\subsection{Esclave}
L'esclave peut effectuer l'action suivante :
\begin{itemize}
        \item Vérifier la réception d'un message à chaque tour de boucle.
\end{itemize}

\section{Participants}
\begin{itemize}
        \item \textbf{Julien Audibert}, réflexions à la base du protocole de communication et conception de la formation en pyramide, code de base de détection et évitement des obstacles ;
        \item \textbf{Thomas Grillo}, création de la librairie \textit{com}, implémentant le protocole de communication, de la documentation \textit{doxygen}, de l'implémentation en machine à états, co-création des procédures \textit{go} et \textit{suivre}.
        \item \textbf{Rémy Giangrasso}, auteur de la librairie de communication hardware, débuggage partiel de la librairie \textit{com} ;
        \item \textbf{Rémy Garcia}, réflexions sur la conception, rédaction du rapport ;
        \item \textbf{Clément Lavoisier}, code de base de détection et évitement des obstacles ;
        \item \textbf{Mathias Ellapin}, réflexions sur la conception, co-création des procédures \textit{go} et \textit{suivre} ;
\end{itemize}

Autres participants : Arthur Finkelstein, Christophe Fagard, Diana Resmerita, Djoé Denne, Dominique Dib, Kévin Caucheteur, Mehdi El Hajami, Mohamed Limam.
\end{document}
