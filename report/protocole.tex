\documentclass[a4paper,12pt]{article}
\usepackage[T1]{fontenc}
\usepackage[french]{babel}
\usepackage[utf8]{inputenc}
\usepackage{hyperref}
\usepackage{graphicx}
\usepackage{tkz-graph}
\usepackage{multirow}
\usepackage{array}
\newcolumntype{P}[1]{>{\centering\arraybackslash}p{#1}}
\newcolumntype{M}[1]{>{\centering\arraybackslash}m{#1}}
\usepackage{tikz}
\usetikzlibrary{arrows,calc,positioning}
\tikzstyle{intt}=[draw,text centered,minimum size=6em,text width=5.25cm,text height=0.34cm]
\tikzstyle{intl}=[draw,text centered,minimum size=2em,text width=2.75cm,text height=0.34cm]
\tikzstyle{int}=[draw,minimum size=2.5em,text centered,text width=3.5cm]
\tikzstyle{intg}=[draw,minimum size=3em,text centered,text width=6.cm]
\tikzstyle{sum}=[draw,shape=circle,inner sep=2pt,text centered,node distance=3.5cm]
\tikzstyle{summ}=[drawshape=circle,inner sep=4pt,text centered,node distance=3.cm]

\title{%
        Protocole de déplacement pour robots à base d'Arduino\\
        \large Complexité
}
\author{%
        M1 IFI/RIF
}


\begin{document}
\maketitle

%\tableofcontents
\begin{abstract}
        Ce document décrit le protocole conçu pour contrôler la formation et le déplacement des robots à base d'Arduino crées durant le cours de complexité.
\end{abstract}

\newpage

\section{Protocole}\label{protocole}
\subsection{Initialisation}
La phase d'initialisation, à l'allumage des robots, se déroule en trois étapes :
\begin{itemize}
        \item Tirage d'un identifiant unique ;
        \item Demander s'il existe déjà un chef, s'il n'y en a pas, le robot se considère comme étant le chef ;
        \item Envoyer son identifiant unique au chef ;
        \item \textit{Facultatif} : Vérifier s'il existe plusieurs chefs, si oui celui avec l'identifiant le plus grand garde son statut de chef.
\end{itemize}
Pour finir le chef envoie à chaque robot une paire de nombres composée de l'identifiant du robot et de sa position dans la formation.
\subsection{Formation}
Les robots se positionnent en triangle, avec le chef en tête et deux esclaves derrière lui.\\
Chaque esclave est suivit par deux esclaves jusqu'à la fin de la formation.\\
Les robots sont séparés par une distance de $50 \pm 10\%$ cm.
\begin{figure}[!h]
        \centering
        \begin{tikzpicture}
                \GraphInit[vstyle=Empty]
                \Vertex[x=1, y=-2]{C}
                \Vertex[x=0,y=-3]{E1}
                \Vertex[x=2,y=-3]{E2}
                \Vertex[x=-1,y=-4]{E3}
                \Vertex[x=1,y=-4]{E4}
                \Vertex[x=3,y=-4]{E5}
                \Vertex[x=-2,y=-5]{E6}
                \Vertex[x=0,y=-5]{E7}
                \Vertex[x=2,y=-5]{E8}
                \Vertex[x=4,y=-5]{E9}
                \Edges(E1,C,E2)
                \Edges(E3,E1,E4,E2,E5)
                \Edges(E6,E3,E7,E4,E8,E5,E9)
        \end{tikzpicture}
        \caption{Formation avec dix robots}\label{fig:Form}
\end{figure}

\subsection{Mouvement}
Lors du déplacement, le robot avec le rôle de chef respecte les conditions suivantes :
\begin{itemize}
        \item Avancer en ligne droite ;
        \item S'il détecte un obstacle, l'éviter ;
        \item Communiquer ses déplacements aux autres robots.
\end{itemize}
Le reste des robots obéissent aux règles ci-dessous :
\begin{itemize}
        \item Écouter les ordres de déplacements du chef ;
        \item Avancer en fonction de l'ordre reçu et de leur position dans la formation
\end{itemize}

\subsection{Format de communication}

\begin{figure}[!h]
        \centering
        \begin{tabular}{|M{1cm}|M{1cm}|lll}
                \cline{1-2}
                0 & 0 &  &  &  \\ \cline{1-2}
                1 & len &  &  &  \\ \cline{1-2}
                \multicolumn{2}{|c|}{CRC} &  &  &  \\ \cline{1-2}
                \multicolumn{2}{|c|}{\multirow{2}{*}{\ldots}} &  &  &  \\
                \multicolumn{2}{|c|}{} &  &  &  \\ \cline{1-2}
                end & end &  &  &  \\ \cline{1-2}
        \end{tabular}
        \caption{Modèle de message}\label{fig:Mesg}
\end{figure}
Le format de communication se présente comme une trame qui possède une partie fixe et variable avec les éléments suivants :
\begin{itemize}
        \item \textit{0 0} : début de la trame ;
        \item \textit{1} : type de trame ;
        \item \textit{len} : taille de la trame ;
        \item \textit{CRC} : Cyclic redundancy check, XOR sur chaque niveau de la trame (hors CRC) ;
        \item \textit{\ldots} : corps du message ;
        \item \textit{end  end} : fin de la trame ;
\end{itemize}
La trame fait une taille de 32 octets.

%- communication reçu envoyé
%- précision des timeout
%- remplacement des CRC par des XOR niveau par niveau
%- changer couleur arbre communication
\newpage
\subsection{Déroulement des communications}
Le schéma des communications commence toujours par un robot qui envoie "chef", ensuite en fonction de la réponse qu'il reçoit, les communications empruntent une branche de l'arbre des communications.
Les cases bleus sont les messages envoyés et les cases rouges sont les messages reçu par le robot.
\begin{figure}[!h]
        \centering
        \begin{tikzpicture}[
                >=latex',
                auto
                ]
                \node [intg] (kp)  [draw=blue] {chef};
                \node [int]  (ki1) [node distance=1.5cm and -1cm,draw=red,below left=of kp] {Id};
                \node [int]  (ki11) [node distance=1.5cm and -1cm,draw=blue,below =of ki1] {UId};
                \node [int]  (ki12) [node distance=1.5cm and -1cm,draw=red,below =of ki11] {(UId,Num)};
                \node [int]  (ki2) [node distance=1.5cm and -1cm,draw=red,below right=of kp,label=right:\& timeout > 10sec] {0};
                eroiejg
                \node [int]  (ki21) [node distance=1.5cm and -1cm,draw=red,below=of ki2] {chef};
                \node [int]  (ki22) [node distance=1.5cm and -1cm,draw=blue,below=of ki21, ] {Id};
                \node [int]  (ki23) [node distance=1.5cm and -1cm,draw=red,below=of ki22] {UId};
                \node [int]  (ki24) [node distance=1.5cm and -1cm,draw=blue,below=of ki23] {(UId,Num)};
                \node [int]  (ki3)  [node distance=1.5cm and -1cm,draw=blue,below left=of ki24] {direction};
                \node [int]  (ki4)  [node distance=1.5cm and -1cm,draw=blue,below right=of ki24] {(vitesse, temps)};
                \draw[->] (kp) -- ($(kp.south)+(0,-0.75)$) -| (ki1);
                \draw[->] (kp) -- ($(kp.south)+(0,-0.75)$) -| (ki2);
                \draw[->] (ki1) -- (ki11);
                \draw[->] (ki11) -- (ki12);
                \draw[->] (ki2) -- (ki21);
                \draw[->] (ki21) -- (ki22);
                \draw[->] (ki22) -- (ki23);
                \draw[->] (ki23) -- (ki24);
                \draw[->] (ki24) -| ($(ki24.east)+(1,1)$) |- (ki21);
                \node [] [node distance=1cm, right=of ki22]{\& timeout > 2min};
                \draw[->] (ki24) -- ($(ki24.south)+(0,-0.75)$) -| (ki3);
                \draw[->] (ki24) -- ($(ki24.south)+(0,-0.75)$) -| (ki4);

                \begin{scope}[
                        node distance=1em and 1em,
                        ]
                        \node[draw=red,right=-8cm of ki24.west, text opacity=0, label=right:message reçu] (H) {sdgr\newline dfsf};
                        \node[draw=blue,below=of H, text opacity=0, label=right:message envoyé] (A1) {sdgr\newline dfsf};
                        \path let
                        \p2=(H.west),
                        \p1=(A1.east)
                        in
                        node[align=center,text width=\x1-\x2,anchor=south west,inner sep=0pt]
                        (title)
                        at ([yshift=5pt]H.north west)
                        {Légende};
                \end{scope}
        \end{tikzpicture}
        \caption{Déroulement des communications du point de vue d'un robot}
        \label{fig:Comm}
\end{figure}

\end{document}
